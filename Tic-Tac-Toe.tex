\documentclass[12pt]{article}

\title{Tic-Tac-Toe with SAT}
\author{William John Holden}
\date{\today}

\usepackage{amsmath,amsthm}
\usepackage{mathtools}
%\usepackage{parskip}
\usepackage{amssymb}
\usepackage[margin=1.0in]{geometry}
\usepackage[hyphens]{url}
\usepackage[hidelinks]{hyperref}
\usepackage{listings}
\lstset{basicstyle=\ttfamily,columns=flexible,frame=single,breaklines=true,linewidth=1\textwidth}

\newcommand{\tictactoe}[9]{\begin{center}\begin{tabular}{ c | c | c }{$#1$} & {$#2$} & {$#3$} \\ \hline {$#4$} & {$#5$} & {$#6$} \\ \hline {$#7$} & {$#8$} & {$#9$} \end{tabular}\end{center}}

\begin{document}

\maketitle

\section{Introduction}

In this exercise we will construct a series of boolean satisfiability clauses to find a winning move for the game Tic-Tac-Toe.
We treat the SAT solver as a ``black box'' and will not worry about how it works.
A SAT solver accepts a \textit{formula} of \textit{clauses} containing boolean \textit{variables} in \textit{conjunctive normal form} (CNF).
An example of a formula in CNF is

\begin{equation}(A \vee \neg B)(C)(C \vee D \vee E)(B \vee \neg C).\end{equation}

This formula is \textit{satisfiable} with the literals $A=T$, $B=T$, $C=T$, $D=F$, and $E=F$.
This is not the only satisfying assignment.
$D$ and $F$ may each take the values $T$ or $F$, but we are constrained in $C=T$.
If $C=F$ then the clause $(C)$ cannot be satisfied, and any unsatisfied clause falsifies the entire formula.

Some formulas cannot be satisfied under any assignment of literals to variables. An obvious example is $(A)(\neg A)$.
A less obvious example is

\begin{equation}(A \vee \neg B)(B \vee \neg C)(C)(\neg A \vee \neg B \vee \neg C).\end{equation}

In this case, the $(C)$ clause implies $C=T$, which means $B=T$, which then implies $A=T$, but now the final clause is $(F \vee F \vee F)$ which falsifies the formula.

A computer program for finding a set of satisfying literals, or proving that no such assignment exists, is called a \textit{SAT solver}.
Though the boolean satisfiability problem is NP-complete, state of the art SAT solvers are very efficient in practice.
Martin Horenovsky has a very nice article at \url{https://codingnest.com/modern-sat-solvers-fast-neat-underused-part-1-of-n/} on this subject.
I originally learned of the satisfiability problem and SAT solvers attending a course on NP-Complete Problems from Alexander Kulikov (\url{https://www.edx.org/course/np-complete-problems-uc-san-diegox-algs203x}).

Others have successfully solved Sudoku puzzles using SAT solvers.
In this exercise, we will do the same for tic-tac-toe.
Of course, tic-tac-toe is much less computationally demanding than Sudoku.
Tic-tac-toe has a smaller board ($3 \times 3$ instead of $9 \times 9$) and it has fewer constraints.
If we consider the size of the tic-tac-toe puzzle to be $n$ for dimensions $\sqrt{n} \textrm{ rows} \times \sqrt{n} \textrm{ columns} = x \times y$ then we can iterate over every row and column to find a winning move in

\begin{align}O(x \times y) + O(y \times x ) + O(2 \textrm{ diagonals} \times \sqrt{n}) &= \\
O(xy)+O(xy)+O(2\sqrt{n})
&=O(n).
\end{align}

Using an exponential time tool for a linear time problem may seem crazy, but the goal here is to learn to \textit{reduce} problems to SAT.

\section{Reduction to SAT}

\subsection{Variables}

We need to express the positions of our tic-tac-toe board as boolean variables. This is pretty easy. We assign a variable letter to each position in reading order:

\tictactoe{a}{b}{c}{d}{e}{f}{g}{h}{i}

There are actually \textit{three} possible states for each position: X, O, and unassigned. However, in this problem we do not need to concern ourselves with the third state.
We will consider a variable to be ``true'' if it is marked wtih X and false otherwise.

\subsection{Contraint 1: Winning Combinations}

There are eight total winning combinations in tic-tac-toe: satisfy any of the three rows, any of the columns, or either of the diagonals.
I should note that the word \textit{or} in the previous sentence is an inclusive or; winning combinations allow for both a row and a column, both a row and a diagonal, and both a column and a diagonal. For example, the assignment

\tictactoe{o}{o}{x}{o}{o}{x}{x}{x}{}

can be won by setting $i=x$.  We can express the eight winning combinations as

\begin{equation}(a \wedge b \wedge c) \vee (d \wedge e \wedge f) \vee (g \wedge h \wedge i)
\vee
(a \wedge d \wedge g) \vee (b \wedge e \wedge h) \vee (c \wedge f \wedge i)
\vee
(a \wedge e \wedge i) \vee (c \wedge e \wedge g).\end{equation}

This long formula is in \textit{disjunctive normal form} (DNF) and not the CNF we want.


\end{document}
